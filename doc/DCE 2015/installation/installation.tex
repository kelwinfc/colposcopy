\documentclass[12pt,letterpaper,onecolumn]{article}
\usepackage[latin1]{inputenc}
\usepackage{amsmath}
\usepackage{amsfonts}
\usepackage{amssymb}
\author{Kelwin Fernandes (kafc@inesctec.pt)}

\title{Installation Guide}

\begin{document}
\maketitle

\section{Linux}

If you are using Linux you should know what to do. You are going to need
\texttt{git}, \texttt{make}, \texttt{C++}, \texttt{python 2.7}, \texttt{numpy}
\texttt{matplotlib} and \texttt{OpenCV}.

\section{Windows}

\subsection{Cygwin}

Install \texttt{Cygwin} (\texttt{cygwin.com/install.html)}in order to have
access to the full set of Linux commands (or install Linux and be happy).

\subsection{Mingw}

Install mingw (\textit{Minimalist GNU for Windows}) to have access to common GNU
commands like make, c++ compiler, among others. Download the latest version from
\texttt{http://www.mingw.org/} and follow the installation steps. For the purpose
of this project you will need
\texttt{mingw-developer-toolkit}, \texttt{mingw32-base}, \texttt{mingw32-gcc-g++}
and \texttt{msys-base}.


Add the environment variables to your system using the \textit{Getting Started}
guide of the mingw website.

\subsection{Git}

Download the \texttt{git} app for Windows \texttt{http://msysgit.github.com}. When
installing, select the option ``Use Git and optional Unix tools from the Windows
Command Prompt''.

Using this you will have access to the latest version of the code. Create an account
on Github and get a copy of the repository
(\texttt{https://github.com/ kelwinfc/colposcopy}).

Create a \texttt{SSH Key} using the guide  \texttt{help.github.com/articles/generating-ssh-keys}.
Using the \texttt{Git Bash}.
\subsection{Python}

Download the 2.7 version of \texttt{python} from \texttt{https://www.python.org/download}.
Follow the steps to effectively install it.

In order to have access to python from everywhere in your computer follow the steps
specified in \texttt{https://docs.python.org/2/using/windows.html} to add the environment
variables to your system (steps 3.3.1-3.3.2).

\subsection{OpenCV}

\subsection{WxPython}

\subsection{Numpy}

\end{document}
